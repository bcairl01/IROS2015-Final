%% Snippets

\subsection{Joint Controller Dynamics}
Since the problem at hand is focused on the control of a legged system implemented with 
position-controlled servos, we define each \ith torque input, $\tau_{i}$ by
\begin{equation}
\tau_{i} = K_{p} ( \theta_{i,r} - \theta_{i} ) 
\label{eq::servo_control_dynamics}
%\caption{Normal form dynamics of free-floating robotic system}
\end{equation}
%%
%%
where $K_{p} \RealMat{4}{4}$ is a diagonal, positive definite proportional-gain matrix,
and $\theta_{i,r} \RealVec{4}$ is a vector of joint reference commands corresponding to the joints of \ith leg.  

\begin{figure}[h!]
		\centering
		\subfigure{\centering\fbox{\includegraphics[scale=0.375]{paper2_V_60mms_nN_50_nL_2_pos.png}}\caption{}}
		\subfigure{\centering\fbox{\includegraphics[scale=0.375]{paper2_V_60mms_nN_50_nL_2_pos.png}}\caption{}}
		\caption{Trot Gait at 60$\frac{mm}{s}$; a) Trunk Orientation; b) NARX-Network Error and Learning Rate}
		\label{fig::results_1_1}
	\end{figure}
	\begin{figure}[h!]
		\centering
		\subfigure{\centering\fbox{\includegraphics[scale=0.375]{paper2_V_60mms_nN_50_nL_2_pos.png}}\caption{}}
		\subfigure{\centering\fbox{\includegraphics[scale=0.375]{paper2_V_60mms_nN_50_nL_2_pos.png}}\caption{}}
		\caption{Trot Gait at 80$\frac{mm}{s}$; a) Trunk Orientation; b) NARX-Network Error and Learning Rate}
		\label{fig::results_2_1}
	\end{figure



}


For example, the trunk 
of a legged system could be used to pitch and roll sensor units, such as camera's and laser distance sensors, during locomotion. 
For these payload to to be effective, trunk trajectories
need to be nearly noise-free so that sensor measurements are not compromised by various vibrations experienced by the robot's body.
The design of the controller presented in this paper is motivated by a desire to perform the aforementioned sensor 
articulation using trunk motion-control on our in-house quadruped system, \emph{BlueFoot} (see Figure \ref{fig::bluefoot}),
as it executes gaits generated by a CPG routine. Performing such a task during gaiting certainly demands a control law which 
can compensate for the rigors of gaited locomotions. 
