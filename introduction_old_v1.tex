%%% Introduction %%%

Decentralized joint control schemes provide convenient frameworks for the design and control 
of legged, mobile robotic platforms. This is especially so when model-based control schemes are rendered
intractable due to difficulties in modeling the effects discontinuities and variable kinematic constraints
without the aid of numerical solvers or carefully-made approximations.

Provided that each joint actuator has adequate control authority, a decentralized control 
architecture allows for greater focus to be placed on the control of high-level, kinematic planning tasks. 
Moreover, an accurate model of system interactions is not always needed because joint tracking errors 
can largely be ignored. 
	% TO DO :	[ source for this statement]
Control tasks which naturally fall into this realm include gaiting and manipulation control performed in 
the robot task-space.

Nonetheless, fully decentralized architectures present obvious difficulties when it comes to controlling
non-actuated robot states, which depend on a combination of cooperative joint control and some knowledge 
of the surrounding environment to achieve a desired behavior. An prime example of this would be the trunk 
control of a legged robot, which results from composite leg configurations their relation to a standing surface.
	% TO DO :	[ picture of quadruped on surface ]
Disturbance rejection and control, in this case, tends to rely on explicit knowledge of the robot's environmental 
and careful trajectory planning for all task-space variables. Additionally, the fidelity  of this task-space control
regime falls prey to under-actuated robot configurations or \emph{weak} joint controller which cannot ensure
adequate joint trajectory tracking.
	% TO DO :	[ picture of quadruped on surface ]

In light of the aforementioned motivating concerns, this paper will focus on the trunk control of quadruped robots 
implemented with decentralized joint controllers, and gaited using open-loop trajectory control routines in the robot
task-space. Typically, open-loop, gaited locomotion routines, such as [...], serve as low-complexity 
	% TO DO :	[ such as... ]
methods mobilizing a quadruped platform. A major drawback of some open-loop gaiting methods it their tendency to disregard
control over the orientation state of the robot's trunk. disregard of dynamical effects, or the necessity for careful tuning to produce
consistent motions, even when the terrain over which the robot is traversing is highly regular. With enough attention 
configuration and careful tuning, these disturbance are not generally problematic for generating effective platform locomotion. 
However, unexpected changes in the state are sure to occur given of the systems body orientation come with their own an array 
of associated practical concerns.

For example, the trunk of the quadruped robot may contain an array of vision sensors (camera, laser range-finder, etc.)
which are rigidly attached to the robot chassis. Furthermore, the system may rely on actuation of the trunk to manipulate
the orientation of these sensors in a controlled fashion. If left uncorrected by, measurements from these 
sensor will surely be effected by disturbances in the trunk's orientation states robot's trunk during gaiting, which propagate
due to their rigid interface with the robot.

These disturbance at hand manifest as a result of natural excitations that occur as a result of instantaneous changes 
in the distribution of applied external forces as the robot's feet make and break contact with the ground.
Furthermore, changes in contact configurations during gaiting extend to the authority of a legged system has over 
its main body, and could cause for changes between fully-actuated and under-actuated modes, depending on the 
number of foot contacts currently being made with a supporting surface. Given the variability of constraints during 
an arbitrary gait sequences, the dynamics which govern these effects becomes hard to model in closed form.

Using a centralized control architecture in which joints are controlled on the level of actuator torques, 
the aforementioned problems could be easily circumvented by scheduling joint compliances during 
known changes in robot configurations. This is less straightforward when the system is implemented
with decentralized servo controllers, which are typically commanded via goal positions.

Given the periodicity of gaited locomotion it is more tractable to estimate these dynamics with an on-line learning mechanism.

Neural networks have long been applied to the control of robotic systems due to their usefulness
in the design both on-line and off-line of compensatory mechanisms. Typically, these neuro-compensator
are utilized to model unknown dynamical effects, such as sudden parametric changes in physical system
configuration or loading. Namely, neural networks solutions provide convenient mechanisms for modeling and 
identifying complex system dynamics or structured disturbance effects that present difficulties when a 
closed form description is necessary. 





