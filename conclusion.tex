%%% Conclusion %%%

In this paper, a neuro-compensator mechanism based on a NARX-model Neural Network was presented to reject angular trunk state disturbances while executing periodic gait sequences. This control technique has been shown to be effective in suppressing platform disturbances, which suggests that it is extendable to control tasks which require steady articulation of a legged robot's main body. Further studies will aim to improve the degree of disturbance-rejection offered by the compensator and will test its effectiveness during a task in which the main body (trunk) is to be articulated over some desired trajectory. We are looking into further improvements of the neural network compensation scheme which reduces dependency on knowledge of the system mass matrix. Another improvement to the scheme will involve the incorporation of automatic compensator activation/deactivation using the current network prediction error. This will prompt the  NARX-NN to re-adapt in the event of environmental changes or larger variations in gaiting. We are also planning on integrating load sensors onto each foot of the BlueFoot platform so that this method can be implemented on the actual system.