%% Abstract

%This paper addresses a method of adaptive body orientation control for quadruped robots mobilized
%using open-loop gait generation mechanisms, such as limit-cycle methods or Central Pattern Generators (CPGs). 
%The controller presented here is based on a feed-forward, inverse dynamics routine adapted to dynamical
%uncertainties using a Nonlinear Autoregressive Neural Network with eXogenous inputs (NARX-NN) -- a recurrent neural 
%network architecture typically utilized for modeling nonlinear difference systems. Here, the NARX-NN is
%utilized to make predictions about the system dynamics, which exhibit periodicity during the execution of 
%a cyclic gait. Learned dynamical estimates are used to cancel disturbances imparted upon angular trunk states. 
%This control method is particularly applicable to tasks which demand steady articulation of the trunk during gaiting, 
%such as changing the orientation of vision sensors rigidly mounted to the robot's main body.
%These state predictions are then combined with known feedback signals to approximate the robot's
%dynamics during gaiting. signals are combined with reference trajectories for each foot through an inverse kinematic routine, which is
% then used to update joint position commands supplied to each independent joint controller.



The objective of this paper is to achieve disturbance rejection and 
constant orientation of the trunk of a multi-legged (here a quadruped) robot. This 
is significant when payloads (such as cameras, optical systems, armaments) are carried by the robot.  The trunk is stabilized 
by the utilization of an on-line learning method to actively correct the open-loop gait
generated by a central pattern generator (CPG) or a limit-cycle method. The learning method is based on
a Nonlinear Autoregressive Neural Network with Exogenous inputs (NARX-NN)-- a recurrent neural network architecture 
typically utilized for modeling nonlinear difference systems. A supervised learning approach is used to train the NARX-NN. The input
to the neural network includes states of the robot legs, trunk attitude and attitude rates, and feet contact forces. The  
neural network is used to generate the total torque imparted on the robot. 
This approach allows on-line learning of the internal forces and disturbances due
to various effects to be estimated/learned with the neural network for implementation in an inverse dynamics/computed torque 
controller. The controller is utilized to achieve a stable trunk (i.e., a constant orientation of the trunk). The efficacy of the
proposed approach is shown in detailed simulation studies of a quadruped robot. 